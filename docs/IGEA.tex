%! Author = Alessandro
%! Date = 09/01/2026

% Preamble
\documentclass[11pt]{article}

% --- PACCHETTI ---
\usepackage[utf8]{inputenc}
\usepackage[T1]{fontenc}
\usepackage[italian]{babel}
\usepackage{graphicx}
\usepackage{geometry}
\usepackage{xcolor}
\usepackage{changepage}
\usepackage{titlesec}
\usepackage{fancyhdr} % Per l'intestazione personalizzata
\usepackage{float}
\usepackage[none]{hyphenat}
\usepackage{parskip}
\usepackage[colorlinks=true,
    linkcolor=black, % L'indice resta nero
    urlcolor=blue,   % I link web diventano blu
    citecolor=black]{hyperref}

% --- GEOMETRIA PAGINA ---
% headheight=40pt serve per fare spazio al logo nell'header
\geometry{top=2.5cm, bottom=2.5cm, left=2.5cm, right=2.5cm, headheight=40pt}

% --- DEFINIZIONE COLORI ---
\definecolor{igeaPurple}{HTML}{AE4FDD}

% --- CONFIGURAZIONE INTESTAZIONE (HEADER) ---
% Definiamo lo stile che apparirà su TUTTE le pagine tranne la copertina
\pagestyle{fancy}
\fancyhf{} % Pulisce header e footer precedenti

% 1. Logo a sinistra (Dip-Inf)
\lhead{\includegraphics[height=1.2cm]{images/dip-Inf.png}}

% 2. Testo a destra (Laurea e Prof. Palomba)
\rhead{\footnotesize
Laurea Triennale in Informatica - Università di Salerno \\
Corso di Fondamenti di Intelligenza Artificiale - Prof. F. Palomba}

% 3. Numero di pagina in basso al centro
\cfoot{\thepage}

% Ridefiniamo lo stile "plain" (usato dall'indice) per avere lo stesso header
\fancypagestyle{plain}{
    \fancyhf{}
    \lhead{\includegraphics[height=1.2cm]{images/dip-Inf.png}}
    \rhead{\footnotesize Laurea Triennale in Informatica - Università di Salerno \\ Corso di Fondamenti di Intelligenza Artificiale - Prof. F. Palomba}
    \cfoot{\thepage}
}

% --- INIZIO DOCUMENTO ---
\begin{document}

    % ==========================================
    %               COPERTINA
    % ==========================================
    % \thispagestyle{empty} Rimuove header, footer e numero pagina SOLO da questa pagina
    \begin{titlepage}
        \thispagestyle{empty}
        \centering

        % Testo Istituzionale (Come nella versione originale)
        \Large UNIVERSITÀ DEGLI STUDI DI SALERNO \\[0.5cm]
        \Large Corso di Laurea in Informatica \\
        \Large Fondamenti di Intelligenza Artificiale \\[1cm]

        % Logo del progetto IGEA
        \includegraphics[width=0.4\textwidth]{images/logo_IGEA} \\[0.5cm]

        % Titolo Progetto
        \Huge \textbf{IGEA}\\[0.5cm]

        % Autori
        \Large
        \textbf{Gruppo di Progetto:} \\[0.5cm]
        Gennaro Pio Albano (Mat. 0512119547) \\[0.3cm]
        Giuseppe Annunziata (Mat. 0512120144) \\[0.3cm]
        Alessandro Bonelli (Mat. 0512119640) \\[0.3cm]
        Samuele Nacchia (Mat. 0512119128) \\[1.5cm]

        % Logo Unisa in basso
        \includegraphics[width=0.2\textwidth]{images/unisa}

        \vfill
        \large Anno Accademico 2025/2026
    \end{titlepage}

    \newpage

    % ==========================================
    %               INDICE
    % ==========================================
    % Da qui in poi appare l'Header con dip-Inf e Prof. Palomba
    \tableofcontents
    \newpage

    % ==========================================
    %               CONTENUTO
    % ==========================================

    \section{Introduzione}

        Il benessere psicologico degli studenti universitari rappresenta una tematica di crescente rilevanza nel
        panorama accademico e sanitario. Il percorso universitario, spesso caratterizzato da elevate pressioni
        performative, transizioni sociali significative e incertezza verso il futuro, costituisce una fase critica
        che può favorire l'insorgenza di disturbi dell'umore, tra cui la depressione.\newline
        Tale disagio, se trascurato, può aggravarsi fino a livelli insostenibili, portando nei casi più drammatici a
        gesti estremi.




    \subsection{Sistema attuale}

        Attualmente, l’individuazione di studenti universitari a rischio di depressione avviene
        principalmente tramite autosegnalazione o osservazioni indirette da parte di docenti e tutor.\newline
        I servizi di supporto psicologico operano in maniera reattiva, intervenendo solo quando il disagio è
        già esplicitamente manifestato. Non sono presenti strumenti automatici di analisi o predizione basati sui dati,
        rendendo difficile un’identificazione precoce e sistematica degli studenti potenzialmente vulnerabili.


    \subsection{Obiettivi}

        Il sistema IGEA - Intelligent Guide for Emotional Assessment è stato progettato con l’obiettivo
        di fornire un supporto proattivo nella rilevazione precoce della depressione tra gli studenti universitari,
        al fine di favorire interventi tempestivi e mirati da parte dell'università e degli psicologi dell'ateneo.
        IGEA è un alleato nel monitoraggio del benessere psicologico, ma non sostituisce il lavoro degli esperti.
        Il sistema non ha l'intento di diagnosticare o curare la depressione, ma piuttosto di identificare segnali di
        rischio che possano indicare uno stato di disagio emotivo o mentale, permettendo una valutazione iniziale
        della salute psicologica degli studenti.


    % ------------------------------------------

    \newpage

    \section{Descrizione agente}
    Il sistema \textit{IGEA} è modellato come un agente intelligente di tipo
    \textbf{classificatore}, progettato per supportare l’individuazione precoce
    di potenziali stati di disagio psicologico negli studenti universitari.
    L’agente analizza le risposte fornite dagli studenti tramite questionari
    psicologici strutturati e produce una valutazione automatica del rischio
    di sintomi depressivi.

    \subsection{Specifica PEAS}
        Di seguito è riportata la descrizione PEAS dell'ambiente operativo in forma tabellare.

        \begin{table}[H]
            \centering
            \renewcommand{\arraystretch}{1.5} % Spaziatura righe
            \begin{tabular}{|l|p{10cm}|}
                \hline
                \textbf{Componente} & \textbf{Descrizione} \\
                \hline
                \textbf{Performance} & La misura di performance del sistema si basa sulla capacità
                                       dell'agente di distinguere correttamente studenti inclini alla
                                       depressione e studenti non inclini, con particolare attenzione
                                       all'identificazione accurata della classe Depressione (True). \\

                \hline
                \textbf{Environment} & L'ambiente consiste negli studenti universitari, i quali
                                       completano un questionario psicologico per valutare il loro benessere emotivo. \\

                \hline
                \textbf{Actuators} & Gli attuatori consistono in un sistema di classificazione che
                                     assegna un'etichetta (\("\)Depresso\("\) o \("\)Non depresso\("\)) e segnala
                                     gli studenti identificati come a rischio con una panoramica chiara dei diversi report.
                                     Questi attuatori permettono di attivare interventi per supportare gli studenti. \\

                \hline
                \textbf{Sensors} & I sensori consistono nelle risposte al questionario psicologico
                                   fornito dagli studenti, che vengono analizzate dal sistema. \\
                \hline
            \end{tabular}
            \caption{Specifica PEAS dell'agente IGEA}\label{tab:table}
        \end{table}


    \subsection{Specifiche dell'ambiente}

        L’ambiente operativo in cui agisce il sistema \textit{IGEA}
        è costituito dall’insieme dei dati relativi agli studenti universitari, provenienti da questionari
        autocompilati. Il modello di machine learning interagisce con tale ambiente analizzando i dati disponibili
        al fine di stimare il livello di rischio di disagio psicologico, con particolare riferimento a stati
        depressivi.

        L’ambiente di IGEA può essere classificato come segue:

        \begin{itemize}
            \item \textbf{Parzialmente osservabile}:
            l’agente non ha accesso diretto allo stato psicologico reale dello studente, ma solo a informazioni
            indirette e parziali, spesso soggettive e rumorose, come risposte a questionari.
            Di conseguenza, lo stato dell’ambiente non è completamente osservabile.

            \item \textbf{Non deterministico}:
            la relazione tra i dati osservabili e il reale stato emotivo dello studente non è deterministica.
            A parità di input possono corrispondere stati psicologici differenti, a causa di fattori esterni
            non completamente modellabili e dell’elevata variabilità individuale.

            \newpage
            \item \textbf{Episodico}:
            ogni valutazione prodotta dal sistema è indipendente dalle precedenti.\newline Il modello analizza le
            osservazioni raccolte in un determinato istante temporale senza mantenere uno stato interno che
            tenga traccia delle valutazioni passate. Ciascun episodio corrisponde pertanto a una singola
            compilazione del questionario, e la decisione non influenza né dipende da episodi futuri.

            \item \textbf{Statico}:
            durante l’elaborazione dei dati relativi a una singola compilazione del questionario, l’ambiente
            non cambia: le percezioni fornite all’agente e lo stato interno considerato restano invariati fino al
            termine dell’analisi. Eventuali variazioni nello stato emotivo dello studente o nei dati disponibili
            si verificano solo tra episodi distinti, e non influenzano il processo decisionale in corso.

            \item \textbf{Discreto}:
            poiché il sistema produce esclusivamente una classificazione binaria e opera su percezioni,
            stati e azioni discreti

            \item \textbf{Singolo agente}:
            il sistema IGEA opera come agente singolo e non interagisce direttamente con altri agenti
            intelligenti. Esso fornisce supporto decisionale a figure umane quali psicologi e servizi di
            supporto universitari, che rimangono responsabili degli interventi finali.
        \end{itemize}

    % ------------------------------------------

    \section{Identificazione dataset }
    Nella fase iniziale del progetto, abbiamo valutato due diverse strategie per l'acquisizione dei
    dati necessari all'addestramento del modello di Machine Learning:
    \begin{enumerate}
        \item \textbf{Creare} un dataset da zero, sottoponendo un questionario ad un campione di studenti;

        \item \textbf{Cercare} sulla rete un dataset già formato,
                    e adeguarlo alle nostre esigenze;

    \end{enumerate}

    Dopo un'analisi comparativa, la prima opzione è stata scartata a causa di limitazioni metodologiche
    critiche. In primo luogo, la raccolta di un numero di campioni statisticamente significativo avrebbe
    richiesto tempi eccessivamente lunghi.

    Tuttavia, l'ostacolo principale è rappresentato dalla \textbf{mancanza di un esperto di dominio}.\newline
    Per addestrare un modello di classificazione accurato, ogni record deve essere etichettato correttamente
    (es. "Depresso: Sì/No"). Senza la supervisione di un professionista in grado di valutare clinicamente
    le risposte dei candidati, il dataset prodotto sarebbe risultato privo di validità scientifica,
    rendendo il modello inaffidabile.

    Di conseguenza, si è deciso di procedere con la seconda soluzione, individuando sulla
    piattaforma \textbf{Kaggle} il dataset: \href{https://www.kaggle.com/datasets/adilshamim8/student-depression-dataset/data}{\textcolor{blue}{\underline{\textit{"Student Depression Dataset"}}}}



    \subsection{Punti di forza del dataset}
    La scelta è ricaduta su questo archivio per i seguenti motivi:
    \begin{itemize}
        \item \textbf{Varietà dei parametri:}
            Include fattori determinanti come la pressione accademica,
            la soddisfazione nello studio, le ore di sonno e la storia clinica familiare.
        \item \textbf{Ampiezza dei dati:} Le migliaia di osservazioni forniscono una base solida
            per l'addestramento, permettendo all'algoritmo di riconoscere pattern complessi con
            maggiore precisione.
    \end{itemize}

    % ------------------------------------------

    \section{Analisi dataset}

    \section{Data preparation}
    Scrivere qui la preparazione...

    \subsection{Data cleaning}
    Scrivere qui il cleaning...

    \subsection{Feature scaling}
    Scrivere qui lo scaling...

    \subsection{Feature engineering}
    Scrivere qui l'engineering...

    \subsection{Data balancing}
    Scrivere qui il bilanciamento...

    % ------------------------------------------

    \section{Modeling}
    Scrivere qui il modeling...

    \subsection{Scelta algoritmo}
    Scrivere qui la scelta...

    \subsection{Implementazione / addestramento}
    Scrivere qui l'implementazione...

    % ------------------------------------------

    \section{Conclusioni}
    Scrivere qui le conclusioni...

    \subsection{Metriche di valutazioni}
    Scrivere qui le metriche...

    \subsection{Valutazioni}
    Scrivere qui le valutazioni...

    \subsection{Considerazioni finali}
    Scrivere qui le considerazioni...

\end{document}