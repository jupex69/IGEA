%! Author = Alessandro
%! Date = 09/01/2026

% Preamble
\documentclass[11pt]{article}

% --- PACCHETTI ---
\usepackage[utf8]{inputenc}
\usepackage[T1]{fontenc}
\usepackage[italian]{babel}
\usepackage{graphicx}
\usepackage{geometry}
\usepackage{xcolor}
\usepackage{changepage}
\usepackage{titlesec}
\usepackage{fancyhdr} % Per l'intestazione personalizzata
\usepackage{float}
\usepackage[none]{hyphenat}
\usepackage{parskip}
\usepackage[colorlinks=true,
    linkcolor=black, % L'indice resta nero
    urlcolor=blue,   % I link web diventano blu
    citecolor=black]{hyperref}
\usepackage{color}

% --- GEOMETRIA PAGINA ---
% headheight=40pt serve per fare spazio al logo nell'header
\geometry{top=2.5cm, bottom=2.5cm, left=2.5cm, right=2.5cm, headheight=40pt}

% --- DEFINIZIONE COLORI ---
\definecolor{igeaPurple}{HTML}{AE4FDD}

% --- CONFIGURAZIONE INTESTAZIONE (HEADER) ---
% Definiamo lo stile che apparirà su TUTTE le pagine tranne la copertina
\pagestyle{fancy}
\fancyhf{} % Pulisce header e footer precedenti

% 1. Logo a sinistra (Dip-Inf)
\lhead{\includegraphics[height=1.2cm]{images/dip-Inf.png}}

% 2. Testo a destra (Laurea e Prof. Palomba)
\rhead{\footnotesize
Laurea Triennale in Informatica - Università di Salerno \\
Corso di Fondamenti di Intelligenza Artificiale - Prof. F. Palomba}

% 3. Numero di pagina in basso al centro
\cfoot{\thepage}

% Ridefiniamo lo stile "plain" (usato dall'indice) per avere lo stesso header
\fancypagestyle{plain}{
    \fancyhf{}
    \lhead{\includegraphics[height=1.2cm]{images/dip-Inf.png}}
    \rhead{\footnotesize Laurea Triennale in Informatica - Università di Salerno \\ Corso di Fondamenti di Intelligenza Artificiale - Prof. F. Palomba}
    \cfoot{\thepage}
}

% --- INIZIO DOCUMENTO ---
\begin{document}

    % ==========================================
    %               COPERTINA
    % ==========================================
    % \thispagestyle{empty} Rimuove header, footer e numero pagina SOLO da questa pagina
    \begin{titlepage}
        \thispagestyle{empty}
        \centering

        % Testo Istituzionale (Come nella versione originale)
        \Large UNIVERSITÀ DEGLI STUDI DI SALERNO \\[0.5cm]
        \Large Corso di Laurea in Informatica \\
        \Large Fondamenti di Intelligenza Artificiale \\[1cm]

        % Logo del progetto IGEA
        \includegraphics[width=0.4\textwidth]{images/logo_IGEA} \\[0.5cm]

        % Titolo Progetto
        \Huge \textbf{IGEA}\\[0.5cm]

        % Autori
        \Large
        \textbf{Gruppo di Progetto:} \\[0.5cm]
        Gennaro Pio Albano (Mat. 0512119547) \\[0.3cm]
        Giuseppe Annunziata (Mat. 0512120144) \\[0.3cm]
        Alessandro Bonelli (Mat. 0512119640) \\[0.3cm]
        Samuele Nacchia (Mat. 0512119128) \\[1.5cm]

        % Logo Unisa in basso
        \includegraphics[width=0.2\textwidth]{images/unisa}

        \vfill
        \large Anno Accademico 2025/2026
    \end{titlepage}

    \newpage

    % ==========================================
    %               INDICE
    % ==========================================
    % Da qui in poi appare l'Header con dip-Inf e Prof. Palomba
    \tableofcontents
    \newpage

    % ==========================================
    %               CONTENUTO
    % ==========================================

    \section{Introduzione}

        Il benessere psicologico degli studenti universitari rappresenta una tematica di crescente rilevanza nel
        panorama accademico e sanitario. Il percorso universitario, spesso caratterizzato da elevate pressioni
        performative, transizioni sociali significative e incertezza verso il futuro, costituisce una fase critica
        che può favorire l'insorgenza di disturbi dell'umore, tra cui la depressione.\newline
        Tale disagio, se trascurato, può aggravarsi fino a livelli insostenibili, portando nei casi più drammatici a
        gesti estremi.




    \subsection{Sistema attuale}

        Attualmente, l’individuazione di studenti universitari a rischio di depressione avviene
        principalmente tramite autosegnalazione o osservazioni indirette da parte di docenti e tutor.\newline
        I servizi di supporto psicologico operano in maniera reattiva, intervenendo solo quando il disagio è
        già esplicitamente manifestato. Non sono presenti strumenti automatici di analisi o predizione basati sui dati,
        rendendo difficile un’identificazione precoce e sistematica degli studenti potenzialmente vulnerabili.


    \subsection{Obiettivi}

        Il sistema IGEA - Intelligent Guide for Emotional Assessment è stato progettato con l’obiettivo
        di fornire un supporto proattivo nella rilevazione precoce della depressione tra gli studenti universitari,
        al fine di favorire interventi tempestivi e mirati da parte dell'università e degli psicologi dell'ateneo.
        IGEA è un alleato nel monitoraggio del benessere psicologico, ma non sostituisce il lavoro degli esperti.
        Il sistema non ha l'intento di diagnosticare o curare la depressione, ma piuttosto di identificare segnali di
        rischio che possano indicare uno stato di disagio emotivo o mentale, permettendo una valutazione iniziale
        della salute psicologica degli studenti.


    % ------------------------------------------

    \newpage

    \section{Descrizione agente}
    Il sistema \textit{IGEA} è modellato come un agente intelligente di tipo
    \textbf{classificatore}, progettato per supportare l’individuazione precoce
    di potenziali stati di disagio psicologico negli studenti universitari.
    L’agente analizza le risposte fornite dagli studenti tramite questionari
    psicologici strutturati e produce una valutazione automatica del rischio
    di sintomi depressivi.

    \subsection{Specifica PEAS}
        Di seguito è riportata la descrizione PEAS dell'ambiente operativo in forma tabellare.

        \begin{table}[H]
            \centering
            \renewcommand{\arraystretch}{1.5} % Spaziatura righe
            \begin{tabular}{|l|p{10cm}|}
                \hline
                \textbf{Componente} & \textbf{Descrizione} \\
                \hline
                \textbf{Performance} & La misura di performance del sistema si basa sulla capacità
                                       dell'agente di distinguere correttamente studenti inclini alla
                                       depressione e studenti non inclini, con particolare attenzione
                                       all'identificazione accurata della classe Depressione (True). \\

                \hline
                \textbf{Environment} & L'ambiente consiste negli studenti universitari, i quali
                                       completano un questionario psicologico per valutare il loro benessere emotivo. \\

                \hline
                \textbf{Actuators} & Gli attuatori consistono in un sistema di classificazione che
                                     assegna un'etichetta (\("\)Depresso\("\) o \("\)Non depresso\("\)) e segnala
                                     gli studenti identificati come a rischio con una panoramica chiara dei diversi report.
                                     Questi attuatori permettono di attivare interventi per supportare gli studenti. \\

                \hline
                \textbf{Sensors} & I sensori consistono nelle risposte al questionario psicologico
                                   fornito dagli studenti, che vengono analizzate dal sistema. \\
                \hline
            \end{tabular}
            \caption{Specifica PEAS dell'agente IGEA}\label{tab:table}
        \end{table}


    \subsection{Specifiche dell'ambiente}

        L’ambiente operativo in cui agisce il sistema \textit{IGEA}
        è costituito dall’insieme dei dati relativi agli studenti universitari, provenienti da questionari
        autocompilati. Il modello di machine learning interagisce con tale ambiente analizzando i dati disponibili
        al fine di stimare il livello di rischio di disagio psicologico, con particolare riferimento a stati
        depressivi.

        L’ambiente di IGEA può essere classificato come segue:

        \begin{itemize}
            \item \textbf{Parzialmente osservabile}:
            l’agente non ha accesso diretto allo stato psicologico reale dello studente, ma solo a informazioni
            indirette e parziali, spesso soggettive e rumorose, come risposte a questionari.
            Di conseguenza, lo stato dell’ambiente non è completamente osservabile.

            \item \textbf{Non deterministico}:
            la relazione tra i dati osservabili e il reale stato emotivo dello studente non è deterministica.
            A parità di input possono corrispondere stati psicologici differenti, a causa di fattori esterni
            non completamente modellabili e dell’elevata variabilità individuale.

            \newpage
            \item \textbf{Episodico}:
            ogni valutazione prodotta dal sistema è indipendente dalle precedenti.\newline Il modello analizza le
            osservazioni raccolte in un determinato istante temporale senza mantenere uno stato interno che
            tenga traccia delle valutazioni passate. Ciascun episodio corrisponde pertanto a una singola
            compilazione del questionario, e la decisione non influenza né dipende da episodi futuri.

            \item \textbf{Statico}:
            durante l’elaborazione dei dati relativi a una singola compilazione del questionario, l’ambiente
            non cambia: le percezioni fornite all’agente e lo stato interno considerato restano invariati fino al
            termine dell’analisi. Eventuali variazioni nello stato emotivo dello studente o nei dati disponibili
            si verificano solo tra episodi distinti, e non influenzano il processo decisionale in corso.

            \item \textbf{Discreto}:
            poiché il sistema produce esclusivamente una classificazione binaria e opera su percezioni,
            stati e azioni discreti

            \item \textbf{Singolo agente}:
            il sistema IGEA opera come agente singolo e non interagisce direttamente con altri agenti
            intelligenti. Esso fornisce supporto decisionale a figure umane quali psicologi e servizi di
            supporto universitari, che rimangono responsabili degli interventi finali.
        \end{itemize}

    % ------------------------------------------

    \section{Individuazione dataset }
    Nella fase iniziale del progetto, abbiamo valutato due diverse strategie per l'acquisizione dei
    dati necessari all'addestramento del modello di Machine Learning:
    \begin{enumerate}
        \item \textbf{Creare} un dataset da zero, sottoponendo un questionario ad un campione di studenti;

        \item \textbf{Cercare} sulla rete un dataset già formato,
                    e adeguarlo alle nostre esigenze;

    \end{enumerate}

    Dopo un'analisi comparativa, la prima opzione è stata scartata a causa di limitazioni metodologiche
    critiche. In primo luogo, la raccolta di un numero di campioni statisticamente significativo avrebbe
    richiesto tempi eccessivamente lunghi.

    Tuttavia, l'ostacolo principale è rappresentato dalla \textbf{mancanza di un esperto di dominio}.\newline
    Per addestrare un modello di classificazione accurato, ogni record deve essere etichettato correttamente
    (es. "Depresso: Sì/No"). Senza la supervisione di un professionista in grado di valutare clinicamente
    le risposte dei candidati, il dataset prodotto sarebbe risultato privo di validità scientifica,
    rendendo il modello inaffidabile.

    Di conseguenza, si è deciso di procedere con la seconda soluzione, individuando sulla
    piattaforma \textbf{Kaggle} il dataset: \href{https://www.kaggle.com/datasets/adilshamim8/student-depression-dataset/data}{\textcolor{blue}{\underline{\textit{"Student Depression Dataset"}}}}



    \subsection{Punti di forza del dataset}
    La scelta è ricaduta su questo archivio per i seguenti motivi:
    \begin{itemize}
        \item \textbf{Varietà dei parametri:}
            Include fattori determinanti come la pressione accademica,
            la soddisfazione nello studio, le ore di sonno e la storia clinica familiare.
        \item \textbf{Ampiezza dei dati:} Le migliaia di osservazioni forniscono una base solida
            per l'addestramento, permettendo all'algoritmo di riconoscere pattern complessi con
            maggiore precisione.
    \end{itemize}

    % ------------------------------------------

    \subsection{Limiti del dataset}
    Tale dataset presenta i seguenti limiti:
    \begin{itemize}

        \item
        \textbf{Provenienza culturale}: Il dataset proviene da studenti indiani, e le esperienze psicologiche e
        comportamentali potrebbero differire da quelle degli studenti italiani, influenzando i risultati.

        \item
        \textbf{Fattori socio-culturali e accademici}: Sebbene il dataset descriva variabili psicologiche e
        comportamentali legate alla vita universitaria, che non sono specifiche di un singolo contesto nazionale,
        le dinamiche socio-culturali e accademiche in India potrebbero comunque influenzare la rilevazione del rischio di depressione.

        \item
        \textbf{Validità limitata}: Senza una validazione su dati italiani, il modello potrebbe non riflettere
        accuratamente il rischio di depressione tra gli studenti italiani.

    \end{itemize}

    Sebbene il dataset quindi sia stato raccolto su studenti indiani, le informazioni disponibili
    descrivono aspetti psicologici e comportamentali legati alla vita universitaria che non sono
    specifici di un singolo contesto nazionale. Il modello viene pertanto utilizzato come studio
    preliminare per valutare la fattibilità di un sistema predittivo del rischio di depressione,
    riconoscendo che una validazione su dati italiani reali sarebbe necessaria per un impiego operativo.



    % ==========================================
%          DATA UNDERSTANDING
% ==========================================
    \section{Data Understanding}

    Prima di procedere con le  fasi di preparazione dei dati, è stata svolta una fase di \textit{data understanding},
    con l’obiettivo di comprendere la struttura del dataset, il significato delle variabili e la loro coerenza con il
    dominio applicativo del sistema.

    In questa fase sono state analizzate tutte le colonne del dataset, identificandone il ruolo e il tipo di
    informazione rappresentata. In particolare, le variabili sono state suddivise nelle seguenti categorie:

    \begin{itemize}
        \item \textbf{Variabili demografiche}: \textit{Gender}, \textit{Age};
        \item \textbf{Variabili psicologiche e autovalutative}: \textit{Academic Pressure}, \textit{Study Satisfaction}, \textit{Job Satisfaction}, \textit{Have you ever had suicidal thoughts?}, \textit{Family History of Mental Illness}, \textit{Financial Stress};
        \item \textbf{Variabili di contesto accademico e personale}: \textit{Degree}, \textit{Profession}, \textit{Work/Study Hours}, \textit{CGPA}, \textit{Sleep Duration}, \textit{Dietary Habits}, \textit{Work Pressure}, \textit{City}.
    \end{itemize}

    È stata inoltre individuata la variabile target del problema di classificazione, identificata nella colonna
    \textit{Depression}, che rappresenta se lo studente è a rischio di depressione.

    \paragraph{Descrizione semantica delle variabili}

    La Tabella~\ref{tab:semantic_features} riporta una breve descrizione semantica delle variabili presenti nel
    dataset, al fine di chiarirne il significato e il ruolo nel contesto del problema affrontato.

    \begin{table}[H]
        \centering
        \begin{tabular}{|l|p{10cm}|}
            \hline
            \textbf{Variabile} & \textbf{Descrizione semantica} \\
            \hline
            Gender & Genere dichiarato dallo studente. \\
            Age & Età dello studente espressa in anni. \\
            Academic Pressure & Livello di pressione percepita legata alle attività accademiche. \\
            Study Satisfaction & Livello di soddisfazione dello studente rispetto allo studio. \\
            Job Satisfaction & Livello di soddisfazione rispetto ad attività lavorative eventualmente svolte. \\
            CGPA & Media accademica riportata su scala numerica. \\
            Work/Study Hours & Numero medio di ore dedicate quotidianamente a studio e lavoro. \\
            Sleep Duration & Durata media del sonno giornaliero dichiarata. \\
            Dietary Habits & Indicatore delle abitudini alimentari dichiarate. \\
            Degree & Livello di istruzione dichiarato dallo studente. \\
            Financial Stress & Livello di stress percepito legato alla situazione finanziaria. \\
            Have you ever had suicidal thoughts? & Indicatore della presenza di pensieri suicidari auto-riferiti. \\
            Family History of Mental Illness & Presenza di precedenti familiari di disturbi mentali. \\
            Profession & Stato occupazionale dichiarato (nel dataset: Student). \\
            City & Città di residenza dichiarata dallo studente. \\
            Depression & Variabile target che indica la presenza o assenza di depressione. \\
            \hline
        \end{tabular}
        \caption{Descrizione semantica delle variabili del dataset}
        \label{tab:semantic_features}
    \end{table}

    Questa fase di data understanding ha consentito di validare la coerenza del dataset rispetto all’obiettivo
    del sistema e di impostare correttamente l’analisi preliminare e le successive fasi di preparazione dei dati,
    senza introdurre modifiche al dataset originale.






    \subsection{Analisi preliminare del dataset}
    L’analisi preliminare del dataset per comprenderne la struttura e le principali caratteristiche statistiche
    si è concentrata su:

    \begin{itemize}
        \item ricerca di valori mancanti in ciascuna colonna del dataset.
        \item calcolo della dipendenza delle feature numeriche e categoriche dalla variabile target \textit{Depression};
        \item analisi delle distribuzioni delle feature numeriche e categoriche per identificare variabili sbilanciate o a bassa variabilità.
    \end{itemize}

    \newpage
    \paragraph{Valori mancanti}
    L’analisi ha evidenziato che non sono presenti valori mancanti.

    \paragraph{Dipendenza dalla variabile target}

    Per le feature numeriche, è stata calcolata la correlazione di Pearson con la variabile target.
    Per le feature categoriche, la dipendenza è stata stimata come la massima differenza
    tra le percentuali di studenti depressi e non depressi all'interno delle categorie di ciascuna variabile.
    In altre parole, indica quanto la presenza di una determinata categoria sia associata allo stato di depressione.

    I risultati principali sono i seguenti:

    \textbf{Correlazione delle feature numeriche con la target \textit{Depression}:}
    \begin{itemize}
        \item Academic Pressure: 0.475
        \item Age: -0.226
        \item Work/Study Hours: 0.209
        \item Study Satisfaction: -0.168
        \item CGPA: 0.022
        \item Job Satisfaction: -0.003
        \item Work Pressure: -0.003
    \end{itemize}

    \textbf{Correlazione delle feature categoriche con la target \textit{Depression}:}
    \begin{itemize}
        \item Profession: 1.000
        \item Financial Stress: 0.626
        \item Have you ever had suicidal thoughts ?: 0.581
        \item Degree: 0.415
        \item Dietary Habits: 0.415
        \item Sleep Duration: 0.290
        \item Family History of Mental Illness: 0.225
        \item Gender: 0.173
    \end{itemize}

    \paragraph{Analisi delle distribuzioni}

    Per ciascuna feature, oltre a valutare la correlazione o la dipendenza dalla target, è stata analizzata la distribuzione dei valori:

    \begin{itemize}
        \item \textbf{Feature numeriche}: sono state calcolate media, deviazione standard, skewness e kurtosis.
        Feature con skew > 1, kurtosis > 5 o deviazione standard molto bassa (<0.01) sono state considerate \textit{anomale},
        poiché potrebbero non fornire informazioni significative al modello.
        \item \textbf{Feature categoriche}: sono state valutate la numerosità delle categorie e la distribuzione percentuale.
        Variabili con una sola categoria o con una categoria dominante (>90\%) sono considerate sbilanciate e poco informative.
    \end{itemize}

    \section{Data preparation}
    La fase di \textit{data preparation} consente di trasformare il dataset grezzo in un insieme di dati
    coerente e idoneo all’addestramento del modello di Machine Learning.

    In questa fase iniziale si procede a una comprensione approfondita dei dati disponibili,
    al fine di valutarne qualità, completezza e rilevanza rispetto al problema di classificazione affrontato.

    L’approccio adottato per la preparazione dei dati si articola nelle seguenti attività principali:
    \begin{itemize}
        \item analisi preliminare del dataset e delle variabili disponibili;
        \item operazioni di \textit{data cleaning} per la rimozione di dati mancanti, incoerenti o irrilevanti;
        \item applicazione di tecniche di \textit{feature scaling} sulle variabili numeriche;
        \item trasformazione e costruzione di nuove feature significative (\textit{feature engineering});
        \item gestione dello sbilanciamento delle classi tramite tecniche di \textit{data balancing}.
    \end{itemize}

    % ==========================================
    %              DATA CLEANING
    % ==========================================

    \subsection{Data cleaning}
    La fase di \textit{data cleaning} è finalizzata a rimuovere elementi potenzialmente dannosi per il processo
    di apprendimento del modello, con l’obiettivo di migliorare la qualità e l’affidabilità complessiva del dataset.

    In primo luogo, sono state eliminate le colonne non informative o non utilizzabili per il nostro studio:

    \begin{itemize}
        \item \textit{id}: attributo identificativo degli studenti, irrilevante per la predizione del rischio di depressione;
        \item \textit{City}: rappresenta un'informazione geografica specifica del contesto indiano
        e quindi non generalizzabile agli studenti italiani.
    \end{itemize}

    Successivamente, sono state rimosse le feature numeriche e categoriche con distribuzioni anomale o bassa dipendenza dalla target:

    \begin{itemize}
        \item \textbf{Numeriche}: Work Pressure, CGPA, Job Satisfaction
        \item \textbf{Categoriali}: Profession
    \end{itemize}

    Al termine della fase di \textit{data cleaning}, il dataset risultante appare coerente,
    privo di incongruenze significative e adeguato per le successive fasi di trasformazione e addestramento del modello.

    \paragraph{Caso della feature CGPA}

    La feature \textit{CGPA} è stata inizialmente considerata per l'eliminazione poichè la correlazione lineare tra CGPA
    e la variabile target risulta praticamente nulla (0.022), indicando che CGPA non presenta una relazione lineare
    significativa con il rischio di depressione nel dataset a disposizione.

    \textit{CGPA} è stata rimossa nella fase preliminare di \textit{data cleaning},
    poiché la sua presenza avrebbe fornito informazioni minime al modello e potenzialmente aumentato il rumore nei dati.

    Va comunque sottolineato che in fasi successive di sperimentazione con modelli più complessi,
    la feature potrebbe essere reinserita per valutare eventuali effetti non lineari o interazioni con altre variabili.

    % ==========================================
    %              FEATURE SCALING
    % ==========================================

    \subsection{Feature scaling}

    % ==========================================
    %              PIPELINE 1
    % ==========================================

    \subsection{Pipeline 1}

    \subsubsection{Feature engineering}

    \subsubsection{Data balancing}


    % ==========================================
    %              PIPELINE 2
    % ==========================================

    \subsection{Pipeline 2}

    \subsubsection{Feature engineering}

    \subsubsection{Data balancing}


% ------------------------------------------

    \section{Modeling}
    Scrivere qui le conclusioni...

    % ==========================================
    %              PIPELINE 1
    % ==========================================
    \subsection{Pipeline 1}

    \subsubsection{Scelta algoritmo}

    \subsubsection{Implementazione/addestramento}


    % ==========================================
    %              PIPELINE 2
    % ==========================================
    \subsection{Pipeline 2}

    \subsubsection{Scelta algoritmo}

    \subsubsection{Implementazione/addestramento}


% ------------------------------------------

    \section{Conclusioni}
    Scrivere qui le conclusioni...

    \subsection{Metriche di valutazioni}
    Scrivere qui le metriche...

    \subsection{Valutazioni}
    Scrivere qui le valutazioni...

    \subsection{Considerazioni finali}
    Scrivere qui le considerazioni...

\end{document}