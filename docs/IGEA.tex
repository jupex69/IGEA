%! Author = Alessandro
%! Date = 09/01/2026

% Preamble
\documentclass[11pt]{article}

% --- PACCHETTI ---
\usepackage[utf8]{inputenc}
\usepackage[T1]{fontenc}
\usepackage[italian]{babel}
\usepackage{graphicx}
\usepackage{geometry}
\usepackage{xcolor}
\usepackage{changepage}
\usepackage{titlesec}
\usepackage{fancyhdr} % Per l'intestazione personalizzata

% --- GEOMETRIA PAGINA ---
% headheight=40pt serve per fare spazio al logo nell'header
\geometry{top=2.5cm, bottom=2.5cm, left=2.5cm, right=2.5cm, headheight=40pt}

% --- DEFINIZIONE COLORI ---
\definecolor{igeaPurple}{HTML}{AE4FDD}

% --- CONFIGURAZIONE LINK (NERI) ---
\usepackage[colorlinks=true, linkcolor=black, urlcolor=black, citecolor=black]{hyperref}

% --- CONFIGURAZIONE INTESTAZIONE (HEADER) ---
% Definiamo lo stile che apparirà su TUTTE le pagine tranne la copertina
\pagestyle{fancy}
\fancyhf{} % Pulisce header e footer precedenti

% 1. Logo a sinistra (Dip-Inf)
\lhead{\includegraphics[height=1.2cm]{images/dip-Inf.png}}

% 2. Testo a destra (Laurea e Prof. Palomba)
\rhead{\footnotesize
Laurea Triennale in Informatica - Università di Salerno \\
Corso di Fondamenti di Intelligenza Artificiale - Prof. F. Palomba}

% 3. Numero di pagina in basso al centro
\cfoot{\thepage}

% Ridefiniamo lo stile "plain" (usato dall'indice) per avere lo stesso header
\fancypagestyle{plain}{
    \fancyhf{}
    \lhead{\includegraphics[height=1.2cm]{images/dip-Inf.png}}
    \rhead{\footnotesize Laurea Triennale in Informatica - Università di Salerno \\ Corso di Fondamenti di Intelligenza Artificiale - Prof. F. Palomba}
    \cfoot{\thepage}
}

% --- INIZIO DOCUMENTO ---
\begin{document}

    % ==========================================
    %               COPERTINA
    % ==========================================
    % \thispagestyle{empty} Rimuove header, footer e numero pagina SOLO da questa pagina
    \begin{titlepage}
        \thispagestyle{empty}
        \centering

        % Testo Istituzionale (Come nella versione originale)
        \Large UNIVERSITÀ DEGLI STUDI DI SALERNO \\[0.5cm]
        \Large Corso di Laurea in Informatica \\
        \Large Fondamenti di Intelligenza Artificiale \\[1cm]

        % Logo del progetto IGEA
        \includegraphics[width=0.4\textwidth]{images/logo_IGEA} \\[0.5cm]

        % Titolo Progetto
        \Huge \textbf{IGEA}\\[0.5cm]

        % Autori
        \Large
        \textbf{Gruppo di Progetto:} \\[0.5cm]
        Gennaro Pio Albano (Mat. 0512119547) \\[0.3cm]
        Giuseppe Annunziata (Mat. 0512120144) \\[0.3cm]
        Alessandro Bonelli (Mat. 0512119640) \\[0.3cm]
        Samuele Nacchia (Mat. 0512119128) \\[1.5cm]

        % Logo Unisa in basso
        \includegraphics[width=0.2\textwidth]{images/unisa}

        \vfill
        \large Anno Accademico 2025/2026
    \end{titlepage}

    \newpage

    % ==========================================
    %               INDICE
    % ==========================================
    % Da qui in poi appare l'Header con dip-Inf e Prof. Palomba
    \tableofcontents
    \newpage

    % ==========================================
    %               CONTENUTO
    % ==========================================

    \section{Introduzione}

        Il benessere psicologico degli studenti universitari rappresenta una tematica di crescente rilevanza nel
        panorama accademico e sanitario. Il percorso universitario, spesso caratterizzato da elevate pressioni
        performative, transizioni sociali significative e incertezza verso il futuro, costituisce una fase critica
        che può favorire l'insorgenza di disturbi dell'umore, tra cui la depressione.
        Tale disagio, se trascurato, può aggravarsi fino a livelli insostenibili, portando nei casi più drammatici a
        gesti estremi.




    \subsection{Sistema attuale}

        Attualmente, l’individuazione di studenti universitari a rischio di depressione avviene
        principalmente tramite autosegnalazione o osservazioni indirette da parte di docenti e tutor.
        I servizi di supporto psicologico operano in maniera reattiva, intervenendo solo quando il disagio è
        già esplicitamente manifestato. Non sono presenti strumenti automatici di analisi o predizione basati sui dati,
        rendendo difficile un’identificazione precoce e sistematica degli studenti potenzialmente vulnerabili.


    \subsection{Obiettivi}

        Il sistema IGEA - Intelligent Guide for Emotional Assessment è stato progettato con l’obiettivo
        di fornire un supporto proattivo nella rilevazione precoce della depressione tra gli studenti universitari,
        al fine di favorire interventi tempestivi e mirati da parte dell'università e degli psicologi dell'ateneo.
        IGEA è un alleato nel monitoraggio del benessere psicologico, ma non sostituisce il lavoro degli esperti.
        Il sistema non ha l'intento di diagnosticare o curare la depressione, ma piuttosto di identificare segnali di
        rischio che possano indicare uno stato di disagio emotivo o mentale, permettendo una valutazione iniziale
        della salute psicologica degli studenti.


    % ------------------------------------------

    \section{Descrizione agente}
    Scrivere qui la descrizione...

    \subsection{Specifica PEAS}
    Scrivere qui la specifica PEAS...

    \subsection{Analisi del problema}
    Scrivere qui l'analisi...

    % ------------------------------------------

    \section{Raccolta e scelta del dataset}
    Scrivere qui la raccolta dati...

    % ------------------------------------------

    \section{Data preparation}
    Scrivere qui la preparazione...

    \subsection{Data cleaning}
    Scrivere qui il cleaning...

    \subsection{Feature scaling}
    Scrivere qui lo scaling...

    \subsection{Feature engineering}
    Scrivere qui l'engineering...

    \subsection{Data balancing}
    Scrivere qui il bilanciamento...

    % ------------------------------------------

    \section{Modeling}
    Scrivere qui il modeling...

    \subsection{Scelta algoritmo}
    Scrivere qui la scelta...

    \subsection{Implementazione / addestramento}
    Scrivere qui l'implementazione...

    % ------------------------------------------

    \section{Conclusioni}
    Scrivere qui le conclusioni...

    \subsection{Metriche di valutazioni}
    Scrivere qui le metriche...

    \subsection{Valutazioni}
    Scrivere qui le valutazioni...

    \subsection{Considerazioni finali}
    Scrivere qui le considerazioni...

\end{document}